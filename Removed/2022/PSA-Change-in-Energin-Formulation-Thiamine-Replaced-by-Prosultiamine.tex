\chapter{PSA: Change in Energin Formulation (Thiamine Replaced by Prosultiamine)}
\tags{Allithiamine, Bioavailability, Energin, Prosultiamine, Thiamine, Thiamine Pyrophosphate, TPP}
\info{THREAD \#48135~\textbar{}~OCTOBER 21, 2022}

\begin{refsection}

Just a quick heads up. We changed the vitamin B$_{1}$ in Energin from the thiamine HCl to the lipophilic analog known as \underLine{\textsb{prosultiamine}}. The latter is an analog of \underLine{\textsb{allithiamine}}, and allithiamine is well-known to forum users for its high bioavailability and ability to raise levels of thiamine pyrophosphate (TPP) in muscles, brains, and other organs while the same has not been shown to occur with intake of thiamin HCl. TPP is also a required co-factor of the enzyme pyruvate dehydrogenase (PDH), which is the rate-limiting step in the oxidative metabolism of glucose. PDH downregulation has been observed in many diseases including diabetes, obesity, cancer, dementia, MELAS, ALS, myopathies, mental disorders, infectious diseases, sepsis, etc.

There are multiple studies on allithiamine posted on the forum and I have posted below some on prosultiamine as well. As such, the presence of porsultiamine in Energin should increase the effects associated with usage of vitamin B$_{1}$. Both prosultiamine and allithiamine are naturally occurring analogs of vitamin B$_{1}$, and present in high concentrations in garlic and are probably responsible for the unique smell of garlic, with Energin now smelling somewhat similar to garlic as well.\textsuperscript{\cite{urlc24bf378}}

\printbibliography[heading=subbibliography]

\end{refsection}