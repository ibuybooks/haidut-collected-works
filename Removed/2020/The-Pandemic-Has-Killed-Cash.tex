\chapter{The Pandemic Has Killed Cash}
\tags{Cash, Coronavirus, COVID-19, Currency, Digitization, Globalization, Pandemic}
\info{THREAD \#36627~\textbar{}~SEPTEMBER 29, 2020}

\begin{refsection}

Actual article title. I don't think the \enquote{pandemic} has fully killed cash as an exchange token but it does seem to have put a serious dent in its usage. We have had various discussions on the forum about the goal of the elite to achieve full \enquote{digitization} of society, targeting health and finances above all else. Peat also talked in a recent podcast about the, now officially advanced by the Fed, goal of digitizing the dollar and maybe even the Fed doing direct banking with the citizens. Similar projects are underway in China and India, and the latter recently nullified large bills in circulation in order to prevent citizens from stocking up on cash.\textsuperscript{\cite{url5129e82a, url81b31497, url99170cb8}}

\begin{tcolorbox}[quote]

\dots{}\underLine{\textsb{The pandemic has been quietly accelerating the steady slog towards a cashless Britain}}. A decade ago, cash accounted for 58 percent of payments\textsuperscript{\cite{url595bd945}} across the UK. Since then, \underLine{\textsb{national cash usage has plummeted---overtaken by digital payments in 2018---to less than a quarter of all transactions last year}}. \underLine{\textsb{Now, COVID-19 has propelled us into a future where notes and coins are even scarcer}}. ATM withdrawals dropped 60 percent\textsuperscript{\cite{urlae506bee}} when lockdown began in March, according to Link, the UK's largest ATM provider. Constituency data obtained from the GMB trade union shows that cash machines are vanishing at a breakneck rate, with an 8.9 percent drop across the country from April to June. \underLine{\textsb{Throughout lockdown}} 9,000 ATMs\textsuperscript{\cite{url0f0199f9}} \underLine{\textsb{disconnected from Link's network at some point}}---a result of public places closing during the peak of the first wave, or neighbouring machines being removed to promote social distancing. As of July, only 33 percent of these had been reconnected. While withdrawal volumes picked up once restrictions eased, figures from as recently as September 20 show \underLine{\textsb{that usage is still 40 percent lower than it was this time in 2019}}.\textsuperscript{\cite{urlb5e96eda}}

\end{tcolorbox}


\printbibliography[heading=subbibliography]

\end{refsection}