\chapter{Mortality Rate of COVID-19 Is 0.5\%, Much Lower Than CDC/WHO Claims}
\tags{CDC, Coronavirus, COVID-19, Exagerrated, Mortality, Rate, Virus, WHO}
\info{THREAD \#33247~\textbar{}~MARCH 15, 2020}

\begin{refsection}

I try very hard to stay away from politically charged discussions but thought I'd share the studies below, which are arguably more reliable than the countless \enquote{professional opinions} thrown left and right by health \enquote{experts} all over mainstream media. The fact that WHO and CDC chose to fuel the panic from the very beginning without having much hard evidence is very much in line with the incompetence (corruption?) we have been seeing from them in regards to other conditions such as CVD, cancer, diabetes, etc.\textsuperscript{\cite{url09fe0bbf}}

\begin{tcolorbox}[quote]

\dots{}Infections and deaths onboard suggest that the disease's true fatality ratio in China is about 0.5 percent,\textsuperscript{\cite{url09fe0bbf}} though that number may vary from place to place, researchers report March 9 in a paper posted at MedRxiv. \underLine{\textsb{That 0.5 percent is far less than the 3.4 percent of confirmed cases that end in death cited by the World Health Organization}}, but troubling nonetheless. \underLine{\textsb{The WHO's number has come under fire because the true number of people infected with the virus worldwide is not known}}.\textsuperscript{\cite{url9188f03b}}

\end{tcolorbox}

And here is another study that found even lower mortality rate (see attachment).

\begin{tcolorbox}[quote]

\dots{}We also found that 51 most recent \underLine{\textsb{crude infection fatality ratio}} (IFR) and time---delay adjusted IFR \underLine{\textsb{is estimated}} 52 \underLine{\textsb{to be 0.04\%}} (95\% CrI: 0.03\%--0.06\%) and 0.12\% (95\%CrI: 0.08--0.17\%), \underLine{\textsb{which is}} 53 \underLine{\textsb{several orders of magnitude smaller than the crude CFR estimated at 4.19\%}}.\textsuperscript{\cite{url9188f03b}}

\end{tcolorbox}

\begin{tcolorbox}[quote]

\dots{}These findings indicate that \underLine{\textsb{the death risk in Wuhan is estimated to be much higher than those in other areas, which is likely explained by hospital-based transmission}}. Indeed, past nosocomial outbreaks have been reported to elevate the CFR associated with MERS and SARS outbreaks, where \underLine{\textsb{inpatients affected by underlying disease or seniors infected in the hospital setting have raised the CFR}} to values as high as 20\% for a MERS outbreak. Public health authorities are interested in quantifying R and CFR to measure the transmission potential and virulence of an infectious disease, especially when emerging/re emerging epidemics occur in order to decide the intensity of the public health response. \underLine{\textsb{In the context of a substantial fraction of unobserved infections due to COVID-19}}, R \underLine{\textsb{estimates derived from the trajectory of infections}} 280 and the IFR \underLine{\textsb{are more realistic indicators compared to estimates derived from observed cases alone}}.\textsuperscript{\cite{url9188f03b}}

\end{tcolorbox}

\printbibliography[heading=subbibliography]

\end{refsection}