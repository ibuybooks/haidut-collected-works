\chapter{YouTube Will Soon Start Banning COVID-19 Contrarian Videos}
\tags{Ban, Censorship, Contrarian, Coronavirus, COVID-19, Pandemic, Social Distancing, Social Isolation, Vaccines, WHO, YouTube}
\info{THREAD \#36855~\textbar{}~OCTOBER 15, 2020}

\begin{refsection}

Full censorship is finally here. YouTube will start banning any videos discussing COVID-19 vaccines that \enquote{\dots{}\underLine{\textsb{contradict consensus from local health authorities or the World Health Organization}}.} The funny thing is that there is hardly any consensus among WHO and the various federal, state and local health authorities on COVID-19 vaccines. Some of them push hard for vaccines and claim the vaccines are completely safe, others are more cautious and point to the H1N1 vaccine fasco, and yet others still just flat out state that they will issue a ruling discouraging vaccine administration altogether until more is known about their risks and there is more data on their long term safety. This way YouTube can always point to some \enquote{health authority} that claims to have the \enquote{correct} position on COVID-19 vaccines and as such your video would be found to be in violation and banned.

Furthermore, YouTube also said they will ban \enquote{borderline content} in regards to COVID-19 vaccines. Yet, they refused to provide examples of such \enquote{borderline content.} but wait, it gets \enquote{better.} \underLine{\textsb{YouTube will also start banning videos that speak against social isolation and social distancing}}, even if the video does not make any claims in regards to vaccines (COVID-19 or otherwise)! I think that last part is truly an abomination and if it survives a legal challenge then it would be a good indication that the legal system is fully null and void\dots{} or in on the conspiracy.

Bottom line---no matter what you say in regards to COVID-19 on their platform, if they don't like it they will ban it. Now, quite a few people have already woken up to the mass censorship that the IT moguls have imposed on the population and are trying to come up with with alternative platforms that value free speech. However, those platforms are also not safe because ISPs can ban hosts/servers/accounts they do not like. A recent high profile example was 8Chan, which while hosting high objectionable content, was not doing anything illegal yet was still banned by their hosting provider and their CDN. Considering we live in a \enquote{pandemic} with many locations imposing states of emergency, I think an ISP ban on a free-speech promoting platform would be much easier to implement than the 8Chan ban a few years ago during \enquote{normal} times.

Let's see how this unfolds. If the COVID-19 content censorship efforts fail at the platform level I would not be surprised if the ISP are quickly recruited to literally pull the plug on the systems hosting defiant platforms.

\begin{tcolorbox}[quote]

\dots{}\underLine{\textsb{The video platform said it would now ban any content with claims about COVID-19 vaccines that contradict consensus from local health authorities or the World Health Organization}}. YouTube said in an email that this would include removing claims that the vaccine will kill people or cause infertility, or that microchips will be implanted in people who receive the vaccine.\textsuperscript{\cite{url5ae80da6}}

\end{tcolorbox}

\begin{tcolorbox}[quote]

\dots{}YouTube says it already \underLine{\textsb{removes content that}} disputes the existence or transmission of COVID-19, promotes medically unsubstantiated methods of treatment, discourages people from seeking medical care or \underLine{\textsb{explicitly disputes health authorities' guidance on self-isolation or social distancing}}\dots{} In its email, YouTube said it had removed over 200,000 videos related to dangerous or misleading COVID-19 information since early February.\textsuperscript{\cite{url5ae80da6}}

\end{tcolorbox}

\begin{tcolorbox}[quote]

\dots{}Andy Pattison, manager of digital solutions at the World Health Organization, told Reuters that the \underLine{\textsb{WHO meets weekly with the policy team at YouTube to discuss content trends and potentially problematic videos}}. Pattison said the WHO was encouraged by YouTube's announcement on coronavirus vaccine misinformation. \underLine{\textsb{The company also said it was limiting the spread of COVID-19 related misinformation on the site, including certain borderline videos about COVID-19 vaccines}}. \underLine{\textsb{A spokesman declined to provide examples of such borderline content}}.\textsuperscript{\cite{url5ae80da6}}

\end{tcolorbox}

\printbibliography[heading=subbibliography]

\end{refsection}