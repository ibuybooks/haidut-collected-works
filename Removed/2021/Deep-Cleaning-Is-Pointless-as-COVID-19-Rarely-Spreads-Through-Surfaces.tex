\chapter{Deep Cleaning Is Pointless as COVID-19 Rarely Spreads Through Surfaces}
\tags{Cleaning, COVID-19, Fraud, Scam, Spread, Surface}
\info{THREAD \#38600~\textbar{}~FEBRUARY 03, 2021}

\begin{refsection}

Yet another study/article that exposes the charade. Some of that was clear from the start, when all sorts of politicians and doctors (another type of politician) kept admonishing the public and businesses to sanitize \underLine{\textsb{everything}} they touch, yet somehow the packages for online orders (which skyrocketed during the pandemic) were exempt from this risk. Touch the cashier or counter at the local store and you may die, but your Amazon/Walmart/Target/etc. packages are totally safe and do not pose a threat to anybody (even poor grandpa that we all try so hard to kill by not wearing a mask /s).


\begin{tcolorbox}[quote]

\dots{}\underLine{\textsb{By May, the WHO and health agencies around the world were recommending that people in ordinary community settings---houses, buses, churches, schools and shops---should clean and disinfect surfaces, especially those that are frequently touched}}. \underLine{\textsb{Disinfectant factories worked around the clock to keep up with heavy demand}}. But Goldman, a microbiologist at Rutgers New Jersey Medical School in Newark, decided to take a closer look at the evidence around fomites. What he found was that \underLine{\textsb{there was little to support the idea that SARS-CoV-2 passes from one person to another through contaminated surfaces}}. \underLine{\textsb{He wrote a pointed commentary for {\textsbit{The Lancet Infectious Diseases}} in July, arguing that surfaces presented relatively little risk of transmitting the virus}}. His conviction has only strengthened since then, and Goldman has long since abandoned the gloves. Many others reached similar conclusions. In fact, the US Centers for Disease Control and Prevention \underLine{\textsb{(CDC) clarified its guidance about surface transmission in May, stating that this route is \enquote{not thought to be the main way the virus spreads.} It now states that transmission through surfaces is \enquote{not thought to be a common way that COVID-19 spreads}}.}\textsuperscript{\cite{url8213hhds}}

\end{tcolorbox}

\printbibliography[heading=subbibliography]

\end{refsection}