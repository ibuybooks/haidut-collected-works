\chapter{Experts: Japan's Lack of COVID-19 Pandemic Due to the Virus \enquote{Mutating Out of Existence}}
\tags{Coronavirus, COVID-19, Delta, Japan, Mutation, Pandemic, SARS-CoV-2, Self-Destruction}
\info{THREAD \#43857~\textbar{}~NOVEMBER 22, 2021}

\begin{refsection}

I had to rub my eyes a few times when I first read this. You can't make this stuff up. Basically, when the narrative does not fit the raw data coming from a specific country, the \enquote{experts} are ready to concoct pure lunacy and serve it up to the Covidian cult for consumption. It may be just me, but I think the \enquote{expert} claim is physically impossible. There is no single source of SARS-CoV-2 that can somehow mutate out of existence. There are (if you believe the data on SARS-CoV-2 virulence) potentially tens of millions of SARS-CoV-2 infections in Japan, and according to the \enquote{experts,} the virus in every single person somehow underwent spontaneous, simultaneous, and identical mutation(s) that made it less robust/resilient and the immune systems of all those infected people simultaneously crushed it. So, there is now basically no Delta variant in Japan, but of course, that does not mean the pandemic is over, people should get vaccinated, keep social distance, etc., etc.

I bet the same spontaneous self-destruction explanation will be used to end the \enquote{pandemic} in each country that reaches the full vaccination goal, adopts the digital certificate/ID, or generally agrees to play to the tune of the global cabal.\textsuperscript{\cite{url91283ghn}}

\begin{tcolorbox}[quote]

\dots{}\underLine{\textsb{why did Japan's fifth and biggest wave of the coronavirus pandemic, driven by the supercontagious delta variant, suddenly come to an abrupt end following a seemingly relentless rise in new infections}}? And what made Japan different from other developed countries that are now seeing a fresh surge in new cases? According to one group of researchers, \underLine{\textsb{the surprising answer may be that the delta variant took care of itself in an act of \enquote{self-extinction}}.}\textsuperscript{\cite{urlasd90u2s}}

\end{tcolorbox}

\printbibliography[heading=subbibliography]

\end{refsection}