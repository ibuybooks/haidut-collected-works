\chapter{CNN: What's the Point of Ever Leaving Home Again?}
\tags{Conspiracy, COVID-19, Digitization, Great Reset, Institutionalization, Isolation, Pandemic}
\info{THREAD \#39111~\textbar{}~MARCH 08, 2021}

\begin{refsection}

Yet another example that this whole \enquote{pandemic} is nothing but a smokescreen for the ushering in of the deliberate digitization and isolation of society. The former for profit, and the latter for security. Except, we are being gaslighted that the \enquote{security} portion is all about protecting us--i.e., who knows what kinds of dangerous germs we may catch when mingling with others.

The sad reality is that the security attained from isolation is for the powers that be. When we don't meet in person, we cannot really organize a resistance. Communication over digital channels is completely controlled/censored at this point, and even if it was useful, effective resistance still requires physical action by large groups of people. Staying at home 24$\times$7 is exactly what allows the powers that be to keep pushing/deploying their antihumanistic technologies (aka \enquote{Great Reset}), while at the same time staying relatively safe from social unrest. I am reminded of a scene from the movie \enquote{Shawshank Redemption.} There was an old felon (Brooks) who had stayed in prison for pretty much his entire life. When he got out of prison on parole, the world had changed so much that he simply could not fathom surviving in it and\dots{} committed suicide shortly after, as a result of constantly living in fear. He had become institutionalized and was better off inside prison than out in the real world. In the words of his friend Red (Morgan Freeman):

\begin{tcolorbox}[quote]

Red: \underLine{\textsb{These walls are funny}}. \underLine{\textsb{First you hate 'em, then you get used to 'em}}. \underLine{\textsb{Enough time passes, you get so you depend on them}}. \underLine{\textsb{That's institutionalized}}.

Heywood: ***t. I could never get like that.

Ernie: Oh yeah? Say that when you been here as long as Brooks has.

Red : ******* right. \underLine{\textsb{They send you here for life, and that's exactly what they take}}. \underLine{\textsb{The part that counts, anyway}}.\textsuperscript{\cite{urlas9u4123}}

\end{tcolorbox}

It seems the powers that be are pursuing exactly such an agenda---institutionalizing us---into our own homes, by creating a fake but reassuring environment inside, while changing the world outside in such ways that we feel alienated and threatened by it and predictably retreat back into the \enquote{safety} of our homes. This is also the main theme of the great movie series \enquote{The Wire}---i.e., everybody living in the city had become institutionalized---even though the series claimed it was an unintended consequence of how society was structured. If the last year has taught us anything, it is that it has always been deliberate!

\begin{tcolorbox}[quote]

\dots{}A year ago, this was our last normal weekend and we didn't even know it. Now, \underLine{\textsb{we're deep into a pandemic, with ever-rising COVID-19-related deaths, a new strain on the loose and news of people still experiencing symptoms months after testing positive. But as more people get vaccinated, it finally seems like there's hope on the horizon}}. \underLine{\textsb{Eventually, someday soon, the joys of normal life will return. And yet, we have to ask: why bother? What's the point of ever leaving home again}}? Getting dinner, grabbing drinks, working out, seeing a movie. All these comforts of middle class existence in the Before Times are gone, and it could be another year until it's completely safe to return. And after months of pandemmy life, those of us who are lucky enough to work from home have gotten really good at this stay-at-home thing. \underLine{\textsb{Bars? Simply overpriced drinks}}. \underLine{\textsb{Concerts? Too many sweaty armpits, way too close to your face}}. \underLine{\textsb{Gyms? A house party for germs}}. Of course, the pandemic has been a tough time for millions of people---as many are out of work and mental health struggles\textsuperscript{\cite{urlbcf12af5}} continue. Here are some ways you can help.\textsuperscript{\cite{urlf1c27d1d}} \underLine{\textsb{but still, maybe introverts were right all along: why go out when you can stay in}}?

\end{tcolorbox}

\printbibliography[heading=subbibliography]

\end{refsection}