\chapter{In Canada, 75\% of Excess Deaths Were Due to Lockdowns, Not COVID-19}
\tags{Canada, COVID-19, Genocide, Lethal, Lockdown, Pointless, Restriction, Young}
\info{THREAD \#42269~\textbar{}~AUGUST 29, 2021}

\begin{refsection}

We have suspected this for a long time, but this is one of the first studies that provides direct evidence that the lethality of lockdowns was/is much higher than the virus itself. Mainstream media shies away from such studies/reports and the only coverage I have seen in \enquote{reputable} mainstream sources is the link below, which reluctantly admits that at least 1/3 of the excess deaths attributed to the pandemic were not due to COVID-19, but to the social restriction measures many govts implemented.

Well, according to the study below (at least in Canada) that 1/3 number turns out to be more like 3/4 (75\%). In other words, the govt of Canada directly killed 4 people (under 65) for each 1 it \enquote{saved.} and that terrible \enquote{performance} is actually giving the govt of Canada a lot of credit as we don't know if those 20\% \enquote{saved} actually survived because of the restriction measures or due to some other reasons/interventions. If other countries (such as Sweden, Russia, South Korea, Vietnam, etc.) are any indication, lack of lockdowns led to even lower deaths in the <65yo crowd, so the available evidence suggests that the govt of Canada directly killed 5,535 people under 65 years of age for no good reason and without any clear mandate to do so under the Constitution of Canada. Perhaps most importantly, the study below strongly corroborates the hypothesis that the \enquote{excess deaths} numbers we get bombarded with every day by mainstream media are almost entirely due to govt interventions and not the virus. As such, there is every reason to oppose lockdowns and social restrictions and none to support them. So, somebody remind me again, why does the govt (in Canada, at least) still exists considering it does nothing of note except directly killing people?\textsuperscript{\cite{url290hsdbas}}

\begin{tcolorbox}[quote]

\dots{}The COVID-19 pandemic continues to affect communities and families in Canada. Beyond deaths attributed to the disease itself, the pandemic could also have indirect consequences leading to an increase or decrease in the number of deaths due to various factors, including delayed medical procedures, increased substance use, or a decline in deaths attributable to other causes, such as influenza. To understand both the direct and indirect consequences of the pandemic, it is important to measure excess mortality, which occurs when there are more deaths during a period of time than what would have been expected for that period. It should be noted that, even without a pandemic, there is always some year-to-year variation in the number of people who die in a given week. This means that the number of expected deaths should fall within a certain range of values. There is evidence of excess mortality when the number of weekly deaths is consistently higher than what is expected, and even more so when numbers exceed the expected range over consecutive weeks. While we sometimes observe excess mortality that is consistent with the number of deaths attributed to COVID-19, data reveal that indirect consequences of the pandemic are also having a significant impact on the number of excess deaths in Canada, particularly among younger Canadians. Based on the newly updated provisional dataset released today from the Canadian Vital Statistics Death Database, from the end of March 2020 to the beginning of April 2021, an estimated 62,203 deaths were reported among Canadians aged 0 to 64. \underLine{\textsb{This represents 5,535 more deaths than expected were there no pandemic, after accounting for changes in the population such as aging}}. \underLine{\textsb{Over the same period, 1,380 COVID-19 deaths have been attributed to the same age group (those younger than 65), suggesting that the excess mortality is, in large part, related to other factors such as increases in the number deaths attributed to causes associated with substance use and misuse, including unintentional (accidental) poisonings and diseases and conditions related to alcohol consumption}}.\textsuperscript{\cite{urlas0du2bd}}

\end{tcolorbox}

\printbibliography[heading=subbibliography]

\end{refsection}