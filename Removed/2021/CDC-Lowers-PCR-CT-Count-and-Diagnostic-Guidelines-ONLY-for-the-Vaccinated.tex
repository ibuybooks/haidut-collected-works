\chapter{CDC Lowers PCR CT Count and Diagnostic Guidelines Only for the Vaccinated}
\tags{CDC, COVID-19, Fraud, Guidelines, Pandemic, PCR, Symptoms, Vaccines, Virus}
\info{THREAD \#40549~\textbar{}~MAY 24, 2021}

\begin{refsection}

And there it is, the fraud lies in plain sight. If there was ever any doubt that the whole thing was a charade and a vaccine-pushing conspiracy, the articles below should dissipate any such remaining doubts. Quietly, and without a single \enquote{honorable mention} in MSM, on May 1, 2021 CDC changed the guidelines for \underLine{\textsb{both}} the PCR CT threshold and the definitions of what constitutes a COVID-19 case\dots{} but only for those who are vaccinated. In other words, the new guidelines make it virtually impossible for a vaccinated person to be labelled as having COVID-19, but for non-vaccinated people the older, looser guidelines (that virtually guarantee a COVID-19 diagnosis) remain in effect. The new guidelines for vaccinated people are to use a PCR CT count of <28 and to not report asymptomatic or mild cases as an actual COVID-19 \enquote{case.} This is laughable, as CDC is actually asking doctors that \underLine{\textsb{a (vaccinated) person with a positive PCR test (even with a count of <28) and with symptoms not be counted as a COVID-19 case}}! For the unvaccinated, the guidelines remain absurd as before and the journalists who wrote one of the articles confirmed that in their state (Kansas) private labs continue to use PCR CT count of as high as 45 and the state lab used to use a count of 45 (which even Fauci recently admitted produces meaningless results) but has now \enquote{generously} lowered the count to 35. This change by CDC comes a few months after WHO changed its guidelines on PCR CT and symptoms as well (and are actually almost identical to the WHO guidelines) except that WHO did not specify who those guidelines should be used for while CDC explicitly states that those new guidelines should only be used for vaccinated people. So, with the stroke of a pen (or keyboard) the CDC ensures that vaccinated people will almost never be found to have COVID-19 (thus ensuring vaccine effectiveness of 99.999\dots{}\%), while new COVID-19 cases will continue to skyrockets in the future (especially in the fall/winter) and will only be found in the unvaccinated. The latter ensures only the unvaccinated will be blamed for any future \enquote{pandemic} and causing the collapse of the \enquote{health} system.\textsuperscript{\cite{urls0av8bns}}

\begin{tcolorbox}[quote]

\begin{quote}

As of May 1, 2021, CDC transitioned from monitoring all reported vaccine breakthrough cases to focus on identifying and investigating only hospitalized or fatal cases due to any cause. This shift will help maximize the quality of the data collected on cases of greatest clinical and public health importance. Previous case counts, which were last updated on April 26, 2021, are available for reference only and will not be updated moving forward.

\end{quote}

\underLine{\textsb{Just like that, being asymptomatic---or having only minor symptoms---will no longer count as a \enquote{Covid case}}} \underLine{\textsb{but only if you've been vaccinated}}. The CDC has put new policies in place which effectively created a tiered system of diagnosis. Meaning, \underLine{\textsb{from now on, unvaccinated people will find it much easier to be diagnosed with Covid19 than vaccinated people}}.\textsuperscript{\cite{urlrfq3ehgit}}

\end{tcolorbox}

\begin{tcolorbox}[quote]

\dots{}As reported by Daniel Horowitz at Blaze Media,\textsuperscript{\cite{url1af6c096}} the \underLine{\textsb{new CDC guidance for \enquote{COVID-19 vaccine breakthrough case investigation}---meaning people who tested positive after getting vaccinated---says PCR tests should be set at 28 CT or lower}}. \underLine{\textsb{The stated reason for the 28 CT maximum is to avoid false positives on people who have been vaccinated, which would discourage acceptance of the vaccines}}. This is another example of \enquote{following the science} only when it suits a political purpose; to wit, \underLine{\textsb{CDC is not recommending the lower threshold for anyone else being tested}}. False positives must be avoided to encourage vaccinations, but false positives to prevent children from attending school or maintain other government restrictions seem OK with CDC. Last summer, the New York Times reported\textsuperscript{\cite{urla507c2a5}} that CTs above 34 almost never detect live virus but most often, dead nucleotides that are not contagious. \underLine{\textsb{The Sentinel found that many private labs in Kansas used thresholds of 38 and 40, and another one in Lenexa potentially at 45}}. \underLine{\textsb{The state lab at the Kansas Department of Health initially used a 42 CT on its most commonly performed test; on January 7, they reduced it to 35}}.\textsuperscript{\cite{urlxcvz89h3}}

\end{tcolorbox}

\begin{tcolorbox}[quote]

\dots{}Horowitz quotes former New York Times reporter Alex Berenson\textsuperscript{\cite{url986ca7eb}} as saying \underLine{\textsb{a standard of 28 CT applied to the general testing regime would preclude as many as 90\% of cases from being recorded}}.\textsuperscript{\cite{urlxcvz89h3}}

\end{tcolorbox}

\printbibliography[heading=subbibliography]

\end{refsection}